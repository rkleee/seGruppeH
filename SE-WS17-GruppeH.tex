\documentclass[a4paper, 10pt]{scrreprt}


\usepackage[ngerman]{babel}
\usepackage[utf8]{inputenc}




\usepackage{hyperref}



\newcommand{\scenario}[5]{
\begin{description}
\item [\uppercase{Initial assumption}:]\textit{#1}
  \item [\uppercase{Normal}:] \textit{#2}
  \item [\uppercase{What can go wrong}:] \textit{#3}
  \item [\uppercase{Other Activities}:] \textit{#4}
  \item [\uppercase{System State on Completion}:] \textit{#5}
\end{description}
}



\newcounter{UReqCounter}
\newcommand{\ureq}[2]{%
  \vspace{0.5em}%
  \stepcounter{UReqCounter}\noindent%
  \textbf{U-REQ \arabic{UReqCounter}:} #1 \textit{Priority: #2}
}

\newcounter{FReqCounter}

\newcommand{\freq}[2]{%
  \vspace{0.5em}%
  \stepcounter{FReqCounter}\noindent%
  \textbf{F-REQ \arabic{FReqCounter}:} #1 \textit{Priority: #2}
}

\newcounter{NFReqCounter}

\newcommand{\nfreq}[2]{%
  \vspace{0.5em}%
  \stepcounter{NFReqCounter}\noindent%
  \textbf{NF-REQ \arabic{NFReqCounter}:} #1 \textit{Priority: #2}
}





\setcounter{secnumdepth}{5}

\setcounter{tocdepth}{10}



%Glossaries

\usepackage{glossaries}
\usepackage{glossaries-extra}

\makenoidxglossaries

%TODO Glossar - beständig ergänzen

%FIXME

\newcommand{\glx}[1]{#1 \gls{#1}}

\newglossaryentry{Admin}{name={Admin},description={
Der Admin (-istrator) in diesem Dokument ist der Leiter und Verwalter von \gls{Experiment}en.}}

\newglossaryentry{Experiment}{name={Experiment},description={
Das Experiment ist der Vorgang in dem \gls{Client}s eine \gls{Frage} beantworten, darüber diskutieren, evtl. ihre Antworten ändern und alles vom Admin ausgewertet wird.} }

\newglossaryentry{Frage}{name={Frage},description={
Die Frage ist die Grundlage eines jeden \gls{Experiment}s.}}
                                                                                   
\newglossaryentry{Client}{name={Client},description={
Clients sind die Personen, die als ProbandInnen an \gls{Experiment}en teilnehmen. Die Clients entsprechen den Benutzern.
Das sind die Personen, die sich registrieren, \gls{Frage}n beantworten und diskutieren.}}

\newglossaryentry{Gruppe}{name={Gruppe},description={
Die Gruppe ist eine Einteilung von Clients nach verschiedenen Charakteristika.}}

\newglossaryentry{Charakteristik}{name={Charakteristik},description={
Charakteristika entsprechen den vom Client angegebenen Profil-Informationen.}}

\newglossaryentry{Diskussion}{name={Diskussion},description={
Diskussionen sind der Austausch zwischen Clients, anhand derer eine eventuelle Beeinflussung gemessen werden kann.}}

\newglossaryentry{Aktivitaetsscore}{name={Aktivitätsscore},description={
Der Aktivitätsscore soll die Maßeinheit sein, in der eine eventuelle Beeinflussung von Clients gemessen wird.}}

\newglossaryentry{Chat}{name={Chat},description={
Der Chat ist ein sogenanntes Instant-Messaging-Programm wie z.B. WhatsApp.
Über den Chat werden die Diskussionen geführt.}}
%\documentclass[a4paper,10pt]{article}





\newglossaryentry{Chatverbindung}{name={Chatverbindung},description={
Die Chatverbindung ist die Kommunikationserlaubnis zwischen zwei Clients.}}


% Document information

\title{Lastenheft}

\author{author1 \and author2 \and author3 \and author4}

\date{\today}


\begin{document}
\maketitle



\tableofcontents


\begin{abstract}
 
Es soll eine datenbankbasierte Android-App entwickelt werden, 
mit deren Hilfe man analysiert, inwiefern Netzwerk-Topologien individuelle Entscheidungen von sog. Clients beeinflussen. 
Ein Client ist ein Teilnehmer bzw. Benutzer der App, der bestimmte Fragen beantwortet und Diskussionen mit anderen Clients führt. 
Diskussionen zwischen Clients sind Chats, über die z.B. auch Bilder oder anderes Informationsmaterial ausgetauscht werden kann. 
Die andere Gruppe von Benutzern sind die Personen, die diese Experimente durchführen (Admins). 
Diese sollen dazu in der Lage sein, die zu beantwortenden Fragen zu definieren und diese für bestimmte Clientgruppen freizugeben. 
Die Freigabe soll dabei für Clients mit bestimmten Merkmalen erfolgen können (z.B. für alle Clients mit Kindern zwischen 20 und 30 Jahren). 
Zur anschließenden Analyse soll ein filterbarer Datenexport bereitgestellt werden.

\end{abstract}


%\chapter{Introduction}

%FIXME Entweder Intro oder Abstract







%\section{Glossar}

\printnoidxglossary[sort=word]

%Ungenutzte Einträge funktioniert leider noch nicht.

%\printunsrtglossaries


\part{User Requirements}

2. Versuch die ganzen Sachen zu vereinzeln und dabei die Sachen, die in der Beschreibung nicht vorkamen raus zu lassen.


\section{Rollen}

\subsection{AdministratorIn}

\ureq{Es soll einen Zugang für Personen, die mit der App \gls{Experiment}e durchführen, geben. Dieser wird als \gls{Admin} bezeichnet.}{A}

\ureq{Ein Admin soll \gls{Chatverbindung}en zwischen \gls{Client}s hinzufügen und löschen können.}{A}

\ureq{Ein Admin soll eigene \gls{Frage}n erstellen, bearbeiten, löschen, für Clients freigeben, für Clients beenden und auswerten können.}{A}

\ureq{Ein Admin soll anderen Admins eine Berechtigung für seine eigenen Fragen vergeben/entfernen können.}{B}

\ureq{Diese Berechtigung umfasst alle administrativen Operationsmöglichkeiten (siehe Punkt 3) des ursprünglichen Eigentümers.}{B}





\subsection{Client}

\ureq{Es soll einen Zugang für Personen geben, die als ProbandInnen an Experimenten teilnehmen möchten. Dieser wird als \gls{Client} bezeichnet.}{A}

\ureq{Im Zuge der Registrierung der Clients soll es eine Profil-Seite mit (tw. Pflicht-) Angaben zu \gls{Charakteristik}a geben.}{A}      

\ureq{Anhang dieser Charakteristika soll eine im Hintergrund ablaufende Zuordnung zu \gls{Gruppe}n stattfinden.}{A}

\ureq{Clients sollen sich dazu entscheiden können, an \gls{Frage}n teilzunehmen.}{A}

\ureq{Es soll eine \gls{Chat}funktion für die Kommunikation zwischen zwei Clients geben.}{A}

\ureq{Die Chatfunktion soll nur mit freigegebenen Clients nutzbar sein.}{B} 

\ureq{Anhand der abgeschlossenen Experimente und der \gls{Diskussion}en eines jeden Clients soll im Hintergrund ein \gls{Aktivitaetsscore} errechnet werden.}{B}

\ureq{Ein Client soll nur seinen eigenen Aktivitätsscore sehen können.}{C}



\section{Experimente/Fragen}

\ureq{Zu jedem Experiment soll genau eine Frage gehören.}{A}

\ureq{Zu jedem Experiment soll es zeitliche Paramenter geben, wie das vorgesehene Ende des Experiments.}{A}

\ureq{Fragen sollen nur einer \gls{Gruppe} angeboten werden können.}{B}

\ureq{Es sollen Fragen mit verschiedenen auswertbaren Antworttypen (Ja/Nein, Multiple Choice, Bewertung nach Punktesystem bzw. Zahl) möglich sein.}{A}

\ureq{Zu jedem Experiment soll es (vom Admin zugefügte) Informationsmaterialien geben, die den Clients direkt oder als Download zur Verfügung stehen.}{A}




\section{Chat}

\ureq{Der \gls{Chat} soll (Text-)Nachrichten in Echtzeit übermitteln.}{A}

\ureq{Der Chat soll auch Bilder und Dokumente übermitteln}{B}

\ureq{Der Chat soll auch Videos, Audio-Formate und Sprachnachrichten übermitteln.}{C}

\ureq{Es soll eine Funktion für Admins geben, die \gls{Chatverbindung}en zwischen Clients erstellt (Eingabe: Zur Frage angemeldete Clients, deren Charakteristika, sowie die (feste/maximale/flexible) Anzahl der Chatverbindungen pro Client)}{A}



\section{Aktivitätsscore}

\ureq{Jeder Client soll einen eigenen Aktivitätsscore haben.}{A}

\ureq{Der Aktivitätsscore soll für Admins auswertbar sein.}{A}

\ureq{Der Aktivitätsscore soll anhand folgender Informationen systemseitig errechnet werden: Anzahl \gls{Diskussion}en, Anzahl beantworteter Fragen und Qualität der Antworten.}{A}



\section{Datenspeicherung und -verarbeitung}

\ureq{Die erfassten Daten sollen anonymisiert gespeichert werden.}{B}

\ureq{Der Zugriff auf die Datenbank soll auf Admins beschränkt sein.}{B}

%FIXME Korrekten Befehl für Glossareinträge, die anders geschrieben werden als im Glossar benutzen.
\ureq{Die Datenbank soll über Filter verfügen, damit die Daten einzelner \gls{Experiment}e getrennt abgerufen werden können.}{B}

\ureq{Es soll eine Möglichkeit zum Datenexport geben, um die Daten (mit externen Programmen) auswerten zu können.}{C}




%bis hier entfernen für Endversion

\part{System Requirements}

\chapter{Non-Functional Requirements}

To enter a non-functional requirement, simply use:
\begin{verbatim}
  \nfreq{text}{priority}
\end{verbatim}



\section{Product Requirements}

\section{Usability Requirements}

\section{Efficiency Requirements}

\section{Performance Requirements}


\chapter{Functional Requirements}

To enter a functional requirement, simply use:
\begin{verbatim}
  \freq{text}{priority}
\end{verbatim}


\part{Scenarios}

\section{Scenario for collecting medical history}

To enter a scenario, simply use:

\begin{verbatim}
  \scenario{initial assumption}
{normal}
{what can go wrong}
{other activities}
{system state on completion}
\end{verbatim}

For example

\begin{verbatim}
  \scenario{The patient has...}
{\begin{itemize} 
  \item The nurse searches...
\end{itemize}}
{\begin{itemize}
    \item The patient’s...
\end{itemize}}
{Record may be...}
{User is logged...}  
\end{verbatim}

produces:

\scenario{The patient has seen a medical receptionist who has created
  a record in the system and collected the patient’s personal
  information (name, address, age, etc.). A nurse is logged on to the
  system and is collecting medical history.}
{\begin{itemize} 
  \item The nurse searches for the patient by family name. If there is
    more than one patient with the same surname, the given name (first
    name in English) and date of birth are used to identify the
    patient.
  \item The nurse chooses the menu option to add medical history.
  \item The nurse then follows a series of prompts from the system to
    enter information about consultations elsewhere on mental health
    problems (free text input), existing medical conditions (nurse
    selects conditions from menu), medication currently taken
    (selected from menu), allergies (free text), and home life
    (form). 
  \end{itemize}}
{\begin{itemize}
    \item The patient’s record does not exist or cannot be found. The nurse should create a new record and record personal information. 
    \item Patient conditions or medication are not entered in the
      menu. The nurse should choose the ‘other’ option and enter free
      text describing the condition/medication.
    \item Patient cannot/will not provide information on medical
      history. The nurse should enter free text recording the
      patient’s inability/unwillingness to provide information. The
      system should print the standard exclusion form stating that the
      lack of information may mean that treatment will be limited or
      delayed. This should be signed and handed to the patient.
    \end{itemize}}
{Record may be consulted but not edited by other staff
  while information is being entered.}
{User is logged on. The patient record including medical history is
  entered in the database.}  

\end{document}
%%% Local Variables:
%%% mode: latex
%%% TeX-master: t
%%% End:
