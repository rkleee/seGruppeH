\documentclass[a4paper, 10pt]{scrreprt}


\usepackage[ngerman]{babel}
\usepackage[utf8]{inputenc}




\usepackage{hyperref}



\newcommand{\scenario}[5]{
\begin{description}
\item [\uppercase{Initiale Situation}:]\textit{#1}
  \item [\uppercase{Normal}:] \textit{#2}
  \item [\uppercase{Was schief gehen kann}:] \textit{#3}
  \item [\uppercase{Andere Aktivitäten}:] \textit{#4}
  \item [\uppercase{Zustand des Systems bei Erfolg}:] \textit{#5}
\end{description}
}



\newcounter{UReqCounter}
\newcommand{\ureq}[2]{%
  \vspace{0.5em}%
  \stepcounter{UReqCounter}\noindent%
  \textbf{U-REQ \arabic{UReqCounter}:} #1 \textit{Priority: #2}
}

\newcounter{FReqCounter}

\newcommand{\freq}[3]{%
  \vspace{0.5em}%
  \stepcounter{FReqCounter}\noindent%
  \textbf{F-REQ \arabic{FReqCounter}:} #1
  \textit{U-REQ #2}
  \textit{Priority: #3}
}

\newcounter{NFReqCounter}

\newcommand{\nfreq}[2]{%
  \vspace{0.5em}%
  \stepcounter{NFReqCounter}\noindent%
  \textbf{NF-REQ \arabic{NFReqCounter}:} #1 \textit{Priority: #2}
}





\setcounter{secnumdepth}{5}

\setcounter{tocdepth}{10}



%Glossaries

\usepackage{glossaries}
\usepackage{glossaries-extra}

\makenoidxglossaries

%TODO Glossar - beständig ergänzen

%FIXME

\newcommand{\glx}[1]{#1 \gls{#1}}

\newglossaryentry{Admin}{name={Admin},description={
Der Admin (-istrator) in diesem Dokument ist der Leiter und Verwalter von Experimenten.}}

\newglossaryentry{Experiment}{name={Experiment},description={
Das Experiment ist der Vorgang in dem Clients eine Frage beantworten, darüber diskutieren, evtl. ihre Antworten ändern und alles vom Admin ausgewertet wird.} }

\newglossaryentry{Frage}{name={Frage},description={
Die Frage ist die Grundlage eines jeden Experiments.}}
                                                                                   
\newglossaryentry{Client}{name={Client},description={
Clients sind die Personen, die als ProbandInnen an Experimenten teilnehmen. Die Clients entsprechen den Benutzern.
Das sind die Personen, die sich registrieren, Fragen beantworten und diskutieren.}}

\newglossaryentry{Gruppe}{name={Gruppe},description={
Die Gruppe ist eine Einteilung von Clients nach verschiedenen Charakteristika.}}

\newglossaryentry{Charakteristik}{name={Charakteristik},description={
Charakteristika entsprechen den vom Client angegebenen Profilinformationen}}

\newglossaryentry{Diskussion}{name={Diskussion},description={
Diskussionen sind der Austausch zwischen Clients, anhand derer eine eventuelle Beeinflussung gemessen werden kann.}}

\newglossaryentry{Aktivitaetsscore}{name={Aktivitätsscore},description={
Der Aktivitätsscore soll die Maßeinheit sein, in der eine eventuelle Beeinflussung von Clients gemessen wird.}}

\newglossaryentry{Chat}{name={Chat},description={
Der Chat ist ein sogenanntes Instant-Messaging-Programm wie z.B. WhatsApp.
Über den Chat werden die Diskussionen geführt.}}

\newglossaryentry{Chatverbindung}{name={Chatverbindung},description={
Die Chatverbindung ist die Kommunikationserlaubnis zwischen zwei Clients.}}

\newglossaryentry{App}{name={App},description={Für Clients bereit gestellte Andoid App, die über den Google Play Store/von unserem Laptop aus, weil wir keinen PLay Store Account haben installiert wird.}}


% Document information

\title{Lastenheft}

% \author{Jaqueline K. \and Nicole W. \and Rebecca K. \and Lucas B.}
\author{Gruppe H}

\date{\today}


\begin{document}
\maketitle


\tableofcontents


\begin{abstract}
 
Es soll eine datenbankbasierte Android-App entwickelt werden, 
mit deren Hilfe man analysiert, inwiefern Netzwerk-Topologien individuelle Entscheidungen von sog. Clients beeinflussen. 
Ein Client ist ein Teilnehmer bzw. Benutzer der App, der bestimmte Fragen beantwortet und Diskussionen mit anderen Clients führt. 
Diskussionen zwischen Clients sind Chats, über die z.B. auch Bilder oder anderes Informationsmaterial ausgetauscht werden kann. 
Die andere Gruppe von Benutzern sind die Personen, die diese Experimente durchführen (Admins). 
Diese sollen dazu in der Lage sein, die zu beantwortenden Fragen zu definieren und diese für bestimmte Clientgruppen freizugeben. 
Die Freigabe soll dabei für Clients mit bestimmten Merkmalen erfolgen können (z.B. für alle Clients mit Kindern zwischen 20 und 30 Jahren). 
Zur anschließenden Analyse soll ein filterbarer Datenexport bereitgestellt werden.

\end{abstract}


%\chapter{Introduction}

%FIXME Entweder Intro oder Abstract




\chapter{Introduction}

\section{(Initial) Problem Statement for the Mobile Decision Lab}
Der Nutzen des Systems besteht darin, eine mobile Anwendung für Experimente, betreffend Gruppen-Entscheidungen 

The purpose of the system is to have a mobile lab for experiments concerning groups decisions based on different network topologies. The outcome of experiments shall help understand how network structures impact group decisions. The networks are anonymous connections between participants with the possibility to exchange messages between connected participants.

Networks can be constructed, for instance, scale-free or not, or with certain dimensionalities (everyone has exactly one neighbor, exactly two neighbors, etc.) and in a way that they additionally fulfill certain constraints (e.g., a connection between two participants {\em can only exist} if they share some properties, e.g., they are about the same age or have about the same expertise about the subject matter, or such a connection {\em cannot exist} if certain properties hold, e.g., neighbors have to be in different age groups or they are not allowed to have the same level of expertise).

The message format can be plain text or more structured, like an integer, or an enumeration type (like ``up'' or ``down'' or a chess move or a (legal) sentence like ``guilty'' or ```not guilty''').

There are two types of users of the systems: experimenters (admins) or participants (test persons, subjects). Experimenters define the experiments along with all parameters and constraints. Participants can register and decide whether they want to participate in a given experiment. Before the start of the experiment, participants have to answer a few questions related to the task they should solve. (If no such information is requested by the experimenter in the definition of the experiment, this step can be omitted. Answers to these questions are treated confidentially, as all the other information that is entered by participants of the experiments.) Once the experiment has started, they are provided with a task description, which is the problem to be solved and for which a decision has to be reached towards the end of the ``game'' (i.e., has to be entered by each participant individually), and some supporting information (for instance, some paragraphs of law text for legal problems, or a hint on which supporting information they are allowed to use, e.g., some news website or wikipedia articles).

During one round of deliberation, participants can exchange messages with their (anonymous) neighbors about the problem to be solved, according to their own pace, until the predefined time of the experiment is up or the experimenter decides to close the experiments manually. 

When the time of deliberation has ended, each user is asked to decide about the question he or she was asked to answer in the beginning. The answer will be of a predefined format (not free text).

The nature of the experiment should be such that a ``true'' outcome is known already, but hard or impossible to access by participants (like a known sentence for a known case in the legal setting), or will be known shortly in the future (like the outcome of a football game or a chess move by a professional chess player in a certain game or the loss or gain of a stock on the stock market on the next day).

At the end of the experiment, the outcome of the decision (for instance, the fraction of participants choosing ``guilty'' or ``not guilty'' or the fraction predicting the win or loss of a forthcoming football game) can be communicated to the participants, if the experimenter has chosen this option up-front in the definition of the experiment.

All the results, the individual decisions and the group decisions are stored in a database in anonymized form for future analysis.

An experiment can be defined in a way that it is repeated with the same group of people and the same or, alternatively, a different network topology for a specified number of times. These are parameters specified by the experimenter up-front, before the experiment has started.

The outcome that was known already, but hard to access, or that is only known shortly after the experiment has ended, can also be communicated to the participants, if the experimenter wishes to do so.

\subsection{Experimenter Perspective}
The experimenter (also known as the admin) can define an experiment (for details see above) and access the results. When the experiment is defined, the questions to be answered by the participants can be specified, the task description, the supporting information, parameters specifying the network type, parameters regarding the repetition of experiments, and the duration of the experiment (predefined or stopped manually). 

In the first step, the questions to be asked are defined (e.g., ``Are you older than 40?''). The answers serve as background information on the participants, but can also be used as constraints in the automatic and randomized construction of the network. For example, everyone could be connected only to people who are older or younger. Once the network is constructed, a decision problem is sent to the users (along with some material that may help in the decision). The admin can set a time frame for the deliberation or stop the experiment manually at any time. After the time is up or the experiment is stopped manually, and the decisions are entered by all participants, the experimenter has access to the outcomes in anonymized form.

\subsection{Participant Perspective}
The participant can sign up for the mobile decision lab in an android app. He/she may select an experiment to participate in. The participant then has to answer some preliminary questions. After that the participant is presented with a decision problem along with additional material. The participant studies the material and exchanges information and opinions with other participants he/she is connected to in the participant network. Information exchange only happens between two participants at a time. The message format may be different from experiment to experiment. In the simplest case, it is just plain text. After some time of deliberation (discussion of neighboring participants by exchanging messages), the participant is asked to make a final decision. After the decision is made, the participant may get some feedback on her answer depending on the experiment setup (the experiment could also be set up in a way that the participant does not receive feedback). Depending on the user's participation in the experiment, he/she may receive a number of points after each question. The participant can compare his/her performance with other participants on a global leader board. This scoring system is optional and may be chosen by the experimenter in the specification of the experiment.



%\section{Glossar}

\printnoidxglossary[sort=word]

%Ungenutzte Einträge funktioniert leider noch nicht.

%\printunsrtglossaries


\part{User Requirements}


\chapter{Rollen}

\section{AdministratorIn}

\ureq{Es soll einen Zugang für Personen, die mit der \gls{App} \gls{Experiment}e durch\-führen, geben. Dieser wird als \gls{Admin} bezeichnet.}{A}

\ureq{Ein Admin soll \gls{Chatverbindung}en zwischen \gls{Client}s hinzufügen und löschen können.}{A}

\ureq{Ein Admin soll eigene \gls{Frage}n erstellen, bearbeiten, löschen, für Clients freigeben, für Clients beenden und auswerten können.}{A}

\ureq{Ein Admin soll anderen Admins eine Berechtigung für seine eigenen Fragen vergeben/entfernen können.}{B}

\ureq{Diese Berechtigung umfasst alle administrativen Operationsmöglichkeiten (siehe Punkt 3) des ursprünglichen Eigentümers.}{B}





\section{Client}

\ureq{Es soll einen Zugang für Personen geben, die als ProbandInnen an Experimenten teilnehmen möchten. Dieser wird als \gls{Client} bezeichnet.}{A}

\ureq{Im Zuge der Registrierung der Clients soll es eine Profil-Seite mit (tw. Pflicht-) Angaben zu \gls{Charakteristik}a geben.}{A}

\ureq{Anhang dieser Charakteristika soll eine im Hintergrund ablaufende Zuordnung zu \gls{Gruppe}n stattfinden.}{A}

\ureq{Clients sollen sich dazu entscheiden können, an \gls{Frage}n teilzunehmen.}{A}

\ureq{Es soll eine \gls{Chat}funktion für die Kommunikation zwischen zwei Clients geben.}{A}

\ureq{Die Chatfunktion soll nur mit freigegebenen Clients nutzbar sein.}{A} 

\ureq{Anhand der abgeschlossenen Experimente und der \gls{Diskussion}en eines jeden Clients soll im Hintergrund ein \gls{Aktivitaetsscore} errechnet werden.}{B}

\ureq{Ein Client soll nur seinen eigenen Aktivitätsscore sehen können.}{C}



\chapter{Experimente/Fragen}

\ureq{Zu jedem Experiment soll genau eine Frage gehören.}{A}

\ureq{Zu jedem Experiment soll es zeitliche Paramenter geben, wie das vorgesehene Ende des Experiments.}{A}

\ureq{Fragen sollen vom Admin ausgewählten \gls{Gruppe}n angeboten werden können.}{B}

\ureq{Es sollen Fragen mit verschiedenen auswertbaren Antworttypen (Ja/Nein, Multiple Choice, Bewertung nach Punktesystem bzw. Zahl) möglich sein.}{A}

\ureq{Ein Admin kann zu seinen Experimenten Informationsmaterialien hinzufügen, die den Clients direkt oder als Download zur Verfügung stehen.}{A}




\chapter{Chat}

\ureq{Der \gls{Chat} soll (Text-)Nachrichten in Echtzeit übermitteln.}{A}

\ureq{Der Chat soll auch Bilder und Dokumente übermitteln}{B}

\ureq{Der Chat soll auch Videos, Audio-Formate und Sprachnachrichten über\-mitteln.}{C}

\ureq{Es soll eine Funktion für Admins geben, die \gls{Chatverbindung}en zwischen Clients erstellt (Eingabe: Zur Frage angemeldete Clients, deren Charakteristika, sowie die (feste/maximale/flexible) Anzahl der Chatverbindungen pro Client)}{A}



\chapter{Aktivitätsscore}

\ureq{Jeder Client soll einen eigenen Aktivitätsscore haben.}{A}

\ureq{Der Aktivitätsscore soll für Admins auswertbar sein.}{A}

\ureq{Der Aktivitätsscore soll anhand folgender Informationen systemseitig errechnet werden: Anzahl \gls{Diskussion}en, Anzahl beantworteter Fragen und Qualität der Antworten.}{B}



\chapter{Datenspeicherung und -verarbeitung}

\ureq{Die erfassten Daten sollen anonymisiert gespeichert werden.}{A}

\ureq{Der Zugriff auf die Datenbank soll auf Admins beschränkt sein.}{A}

%FIXME Korrekten Befehl für Glossareinträge, die anders geschrieben werden als im Glossar benutzen.
\ureq{Die Datenbank soll über Filter verfügen, damit die Daten einzelner \gls{Experiment}e getrennt abgerufen werden können.}{B}

\ureq{Es soll eine Möglichkeit zum Datenexport geben, um die Daten (mit externen Programmen) auswerten zu können.}{B}




%bis hier entfernen für Endversion

\part{System Requirements}

\chapter{Non-Functional Requirements}

\section{Product Requirements}
\nfreq{Beispiel1}{A}

\section{Usability Requirements}
\nfreq{Beispiel2}{A}

\section{Efficiency Requirements}
\nfreq{Beispiel3}{A}

\section{Performance Requirements}
\nfreq{Beispiel4}{A}


\chapter{Funktionale Anforderungen}

\section{Technische Anforderungen}

\freq{Wir verwenden eine Postgre-SQL Datenbank.}{1}{A}

\freq{Benutzer können sich (mit E-Mail-Adresse, Benutzername und Passwort) am System registrieren und ihren Account verwalten. Dazu gibt es eine Benutzer-Tabelle.}{27}{A}

\freq{In dieser Tabelle gibt es ein Boolean-Feld über das ein Client sich für den Admin-Status bewerben kann.}{1}{B}

\freq{Die Admins stehen ebenfalls in der gleichen Benutzer-Tabelle und sind über ein Boolean-Feld gekennzeichnet.}{1}{A}

\freq{SysAdmins haben Zugriff auf die gesamte Datenbank und können (markierte) Clients zu Admins befördern. Sie können den Admin-Status auch wieder entfernen.}{2}{A}

\freq{Durch diese Struktur sind Admins auch gleichzeitig Clients.}{3}{A}

\freq{Für Admins können in der Datenbank zusätzliche Informationen gespeichert werden (Name, Grund für die Nutzung des Systems, Universität, Studienfach)}{4}{A}

\freq{Für jeden Client wird in der Datenbank der Benutzername und ein Passwort gespeichert.}{5}{A}

\freq{Die Datenbank ist in der Lage für jede Admin-ID einen Wiederherstellungs-Code zu generieren, über den er ein neues Passwort festlegen kann.}{6}{A}

\freq{Die Datenbank speichert die durch Admins erstellte Experimente in einer Tabelle.}{7}{A}

\freq{In der Experiment-Tabelle gibt es ein Feld zur Eingabe der Experiment-Beschreibung.}{8}{B}

\freq{In der Datenbank existiert eine erweiterbare Tabelle mit Sprachen.}{9}{A}

\freq{Der Admin wählt experimentbezogen eine Sprache aus. Diese wird in der Experiment-Tabelle hinterlegt. (Default: Englisch.)}{9}{A}

\freq{Eine zu den Experimenten verlinkte Tabelle speichert "preliminary questions".}{10}{A}

\freq{In der Experiment-Tabelle kann in einem Feld vom Admin die eigentliche Frage hinterlegt werden.}{11}{A}

\freq{In der Experiment-Tabelle kann in einem Feld vom Admin der Antworttyp eingetragen werden (Multiple-Choice, Datum, Zahl, Boolean)}{11}{A}

\freq{In der Experiment-Tabelle kann in einem Feld (eins je Antworttyp) vom Admin die korrekte angegeben werden.}{12}{A}

\freq{Admins können zu Ihren Fragen Dokumente in bestimmten Formaten (PDF, JPG, PNG, ...) hochladen.}{13}{A}

\freq{In der Datenbank gibt es eine Sequenz-Tabelle, in der eine Abfolge von Experimenten gespeichert werden kann.}{14}{C}

\freq{In der Datenbank gespeicherte Fragen können von Admins kopiert werden, wobei Parameter geändert werden können (preliminary Questions, Dokumente, Beschreibung, ...)}{15}{A}

\freq{Es gibt in der Datenbank eine Compound-Tabelle, die die Netzwerk-Topologie speichert (Clientverbindungen)}{16}{A}

\freq{In der Compound-Tabelle gibt es mehrere Boolean-Felder für das Nachrichtenformat, das vom Admin festgelegt wird.}{17}{A}

\freq{In der Experiment-Tabelle gibt es Felder den Start- und Endzeitpunkt. (vom Admin festgelegt)}{18}{A}

\freq{In der Experiment-Tabelle gibt es ein Feld für die Bedenkzeit (vom Admin festgelegt).}{19}{A}

\freq{In der Experiment-Tabelle gibt es ein Boolean-Feld für die Markierung von inaktiven Experimenten (vom Admin festgelegt)}{20}{A}

\freq{In der Benutzer-Tabelle existiert ein Feld für den Aktivitätsscore.}{21}{B}

\freq{In der Experiment-Tabelle existiert ein Boolean-Feld, das angibt, ob diese Frage den Aktivitätsscore beeinflusst (vom Admin festgelegt).}{21}{B}

\freq{In der Experiment-Tabelle muss vom Admin festgelegt werden können, ob das Ergebnis nur ihm, öffentlich, oder nur bestimmten Usern zur Verfügung steht.}{22}{A}

\freq{Die Datenbank speichert die Antworten in einer Antwort-Tabelle, in der die User-IDs der beantwortenden Personen verschlüsselt sind.}{23}{A}

\freq{Die Datenbank ist in der Lage einen Export bestimmter Daten zur Verfügung zu stellen. Admins können diese herunterladen.}{24}{C}

\freq{Siehe F-REQ 30}{25}{A}

\freq{Admins sind über das System in der Lage, SysAdmins zu kontaktieren (z.B. per Kontaktformular).}{26}{A}

\freq{Es gibt in der Benutzer-Tabelle einen Unique-Key mit Benutzername bzw. E-Mail Adresse.}{28}{A}

\freq{Bei der Registrierung wird ein Benutzer aufgefordert, die Nutzungsbedingungen zu akzeptieren.}{29}{A}

\freq{Die Datenbank ist in der Lage für jede Benutzer-ID einen Wiederherstellungscode zu generieren, über den er ein neues Passwort festlegen kann.}{30}{A}

\freq{Das System stellt ein (Video-)Tutorial für neue Benutzer bereit.}{31}{D}

\freq{Benutzer können das Tutorial überspringen bzw. beenden.}{32}{D}

\freq{Es gibt eine API-Schnittstelle zu anderen Datenbanken (z.B. Facebook), über die Benutzer-Account exportiert bzw. importiert werden können.}{33}{D}

\freq{Es gibt für Clients eine filterbare Übersicht aller Experimente mit deren Beschreibungen.}{34}{A}

\freq{Das System stellt Clients eine Übersicht der Experimente bereit, an denen sie teilnehmen können.}{35}{A}

\freq{Das System stellt Clients eine Übersicht der Experimente bereit, an denen sie aktuell teilnehmen.}{36}{A}

\freq{In der Experiment-Übersicht für Clients soll der Start- und Endzeitpunkt bzw. Status (aktiv / inaktiv).}{37}{B}

\freq{Das System akzeptiert Antworten auf "preliminary Questions" nur von am Experiment teilnehmenden Benutzern.}{38}{A}

\freq{Das System zeigt jedem teilnehmenden Client den Wert der verbleibenden Bedenkzeit an.}{39}{A}

\freq{Das System zeigt den teilnehmenden Clients die eigentliche Frage nur während der Bedenkzeit an.}{40}{A}

\freq{Das System ermöglicht teilnehmenden Clients nach der Bedenkzeit die Eingabe der Antworten.}{41}{A}

\freq{Das System sperrt die Antworteingabe nach dem Endzeitpunkt des Experiments.}{42}{A}

\freq{Das System zeigt jedem teilnehmenden Client nach der Bedenkzeit die verbleibende Antwortzeit bis zum Endzeitpunkt an.}{43}{B}

\freq{Teilnehmende Clients sollen in der Lage sein, die hochgeladenen Dokumente herunterzuladen.}{44}{A}

\freq{Teilnehmende Clients sollen (falls vom Admin so definiert) in der Lage sein, nach dem Endzeitpunkt die korrekte Antwort einer Frage zu sehen.}{45}{B}

\freq{Das System soll über verschiedene Einstellungen verfügen.}{46}{B}

\freq{Diese Einstellungen sollen individuell von Benutzern konfiguriert werden können.}{47}{D}

\freq{Clients sind über das System in der Lage, SysAdmins zu kontaktieren (z.B. per Kontaktformular).}{48}{C}

\freq{In den individuellen Einstellungen können Clients ihre bevorzugte Sprache konfigurieren.}{49}{A}

\freq{Wenn eine Clientverbindung im System besteht, können die beteiligten Clients gegenseitig ihre Benutzernamen sehen.}{50}{A}

\freq{Wenn eine Clientverbindung im System besteht, können beteiligte Clients während der Bedenkzeit Nachrichten austauschen.}{51}{A}

\freq{Clients sind über das System in der Lage, Admins zu kontaktieren (z.B. per Kontaktformular).}{52}{B}

\freq{In der Compound-Tabelle existiert ein Boolean-Feld, über das die Clientverbindung stummgeschaltet werden kann.}{53}{B}

\freq{Das System stellt eine geeignete Filterfunktion für unangemessene Ausdrücken in Chats zur Verfügung.}{54}{B}

\freq{Das System ermöglicht die gleichzeitige Auslieferung einer Nachricht an mehrere Clients.}{55}{B}

\freq{Das System stellt eine globale Rangliste aller Clients mit deren Aktivitätsscore zu Verfügung (in absteigendender Reihenfolge).}{56}{C}

\freq{Das System zeigt Clients ihre Position in der globalen Rangliste an.}{57}{C}

\freq{Das System soll eine Weboberfläche zur Verfügung stellen.}{58}{A}

\freq{Das System soll über eine mobile App benutzbar sein.}{59}{A}

\freq{Das System soll ethischen Leitlinien unterliegen.}{60}{A}

\freq{Die Datenbank erfüllt zeitgemäße Sicherheitsstandards}{61}{A}


\part{Szenarios}

\chapter{Szenario für die Erstellung eines Experiments}

\scenario{Ein Admin möchte ein \gls{Experiment} erstellen: Wie lange muss ein verurteilter Täter in Haft? Dazu lädt er eine PDF-Datei mit Informationen zum Tathergang und dem Urteil hoch und ordnet eine \gls{Gruppe} dieser Frage zu.}
{\begin{itemize} 
  \item Der Admin benutzt das Experiment-Erstellungs-Interface und gibt die Frage, die Antwortmöglichkeit als Kommazahl und eine Fragebeschreibung (z.B. Bitte zugeordnetes PDF beachten. Kommazahlen als Antwort sind erlaubt.) ein.
  \item Der Admin gibt außerdem einen Zeitraum für das Experiment, sowie die Dauer für Clients, eine Antwort abgeben zu können, an.
  \item Der Admin lädt das Info-PDF hoch.
  \item Der Admin sucht eine Gruppe aus und ordnet diese zu. (z.B. Jura-Studenten)
  \end{itemize}}
{\begin{itemize}
    \item Die Eingaben (Fragetext und Fragebeschreibung) könnten zu lang sein.
    \item Das PDF könnte zu groß oder im falschen Format sein.
    \item ...
    \end{itemize}}
{Eingabefelder in ihrer Länge beschränken. Checks einbauen, die das PDF auf Größe (z.B. <5 MB) und auf das Format prüfen.}
{User ist eingeloggt. Die Frage ist erstellt und wird mit PDF und allen weiteren vom Admin eingegebenen Infos den ausgewählten Clients zur Teilnahme bereitgestellt.}

\chapter{Szenario für die Registrierung als Client}

\scenario{Eine Person möchte sich über die \gls{App} als \gls{Client} registrieren.}
{\begin{itemize} 
  \item Die Person installiert die App über den Google Play Store.
  \item Sie startet die App, die sich mit dem Server verbindet und einen Registrierungsbildschirm anzeigt.
  \item Sie wird aufgefordert die für die Registrierung erforderlichen Daten einzugeben. Sie gibt diese ein und bestätigt die Übermittlung an den Server.
  \item Der Server speichert die Eingaben in der Datenbank und meldet die erfolgreiche Registrierung zurück.
  \end{itemize}}
{\begin{itemize}
    \item Die App ist für das verwendete OS nicht verfügbar.
    \item Die App ist zwar verfügbar, läuft aber auf dem verwendeten System nicht.
    \item Die App läuft unbrauchbar langsam.
    \item Es ist keine Verbindung zum Server möglich.
    \item Der Server ist überlastet.
    \item Die Internetverbindung wird während der Kommunikation mit dem Server unterbrochen.
    \item Ein frei wählbarer Bezeichner, der als Schlüssel für die Datenbank verwendet wird, ist schon belegt.
    \item Der User scheitert an der Eingabe der Daten.
    \item Die Übermittlung der Daten ist unvollständig/wird abgebrochen.
    \item Der Server verarbeitet die Eingabedaten fehlerhaft.
    \item Die erfolgreiche Registrierung mit Eintrag in die Datenbank kann nicht übermittelt werden, weil die Internetverbindung zwischenzeitlich abgebrochen ist.
    \item Die App wurde zwischenzeitlich beendet, die erfolgreiche Registrierung kann nicht übermittelt werden.
    \end{itemize}}
{Eine andere Person versucht den gleichen Schlüssel zur Registierung zu benutzen, andere Zugriffe auf den Server und die Datenbank, Datenbankeinträge könnten abgefragt werden, bevor sie fertig sind.}
{Neuer User ist eingeloggt, Eintrag in der Datenbank ist angelegt, Server und Client sind bereit zur Kommunikation.}




\end{document}
%%% Local Variables:
%%% mode: latex
%%% TeX-master: t
%%% End:
