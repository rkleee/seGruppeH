\documentclass{report}


\usepackage{pgfgantt}
\usepackage{pdflscape}


\begin{document}

\chapter{Lustiges Frage-Antwort-Spiel mit pseudowissenschaftlichem Flair}


\section*{Aufgabenbereiche (Work Packages)}

\begin{enumerate}
\item [WP1] Projektplan und Arbeitsaufteilung (Gruppe H)
\item [WP2] Datenbankdesign (Gruppe H)
\item [WP3] Implementierung der App (Gruppe I)
\item [WP4] Implementierung der Weboberfläche (Gruppe G)
\item [WP5] Funktions- und Kompatibilitätstest (Gruppe J)
\item [WP6] Marketing und Verteilung (Gruppe K)
\item [WP7] Koordination und Management (Alle irgendwie)
\end{enumerate}


\subsection*{WP1 Projektplan und Arbeitsaufteilung}

\subsubsection{Ziele:} Alle beteiligten Entwickler wissen, was die eigenen Aufgaben sind und wie diese (als Module) das fertige Gesamtprojekt ergeben.
\subsubsection{Beschreibung der Arbeit: [Wochen: 1-1]} Erstellung einer strukturierten Übersicht, die jedem Entwickler(-Team) die Aufgaben (mit Zeitplan) zuweist.

\subsection*{Aufgaben (Tasks)}

\begin{enumerate}
\item [T1.1] Erstellung des Projektplans (Alle Gruppen) (W1, W1) (1 W): \emph{ Aufteilung des Gesamtprojekts in Unterpunkte (Module), die auf einzelne Entwickler(-Teams) aufgeteilt werden.}
\item [T1.2] Erstellung Gantt-Chart (Alle Gruppen) (W1, W1) (1 W) : \emph{ Zeitliche Übersicht über alle Arbeitsprozesse (Tasks) im gesamten Projektzeitraum. Meilensteine kennzeichnen die Fertigstellung wichtiger Projektphasen.}
\item [T1.3] Erstellung Group Allocation Table (Alle Gruppen) (W1, W1) (1 W): \emph{ Beinhaltet alle beteiligten Entwickler(-Teams) und stellt dar, wer zu welchem Zeitpunkt an welchem Task arbeiten soll.}
\end{enumerate}

\subsection*{WP2 Datenbankdesign}

\subsubsection{Ziele:} Sinnvolle Strukturierung der für die Anwendung erforderlichen Daten in verknüpften Tabellen. Dadurch ist eine einfache Auswertung relevanter Daten möglich.
\subsubsection{Beschreibung der Arbeit: [Wochen: 2-XX]} Auswahl einer zu den Anforderungen passenden Datenbank (z.B. MySQL, Access, PostgreSQL, MongoDB, CouchDB). Planung und Erstellung der Tabellen mit zugehörigen Verknüpfungen.


\subsection*{Tasks}

\begin{enumerate}
\item [T2.1] Auswahl einer geeigneten Datenbank (H) (W2, W3) (2 W): \emph{ Analyse der Anforderungen an die Datenbank.}
\item [T2.2] Konzept für notwendige Tabellen (inkl. Normalisierung) (H) (W6, W7) (2 W): \emph{ Übersicht über alle notwendigen Informationen und welche Tabellen wie und womit zusammenhängen.}
\item [T2.3] Erstellung der Tabellen (H) (W17, W18) (2 W): \emph{ Anhand der Requirements werden Tabellen erstellt und sinnvolle Zusammenhänge geschaffen (Fremdschlüssel, Eindeutige Keys, ...)}
\item [T2.4] Dokumentation (H) (W81, W18) (1 W): \emph{ Dokumentation aller Tabllen, sodass jeder beteiligte Entwickler weiß, wie die Datenbank zu behandeln ist (auch wichtig für die Weiterentwicklung).}
\item [T2.5] Erweiterung der Datenbank je nach Bedarf (H) (W19, W++) (X W): \emph{ Die Datenbank muss stetig aktualisiert und erweitert werden, falls neue Anforderungen entstehen.}
\item [T2.6] Datenbankpflege (H) (W19, W++) (X W) : \emph{ Stetige Aktualisierung der Datenbank-Version (z.B. wegen Sicherheitslücken)}
\end{enumerate}

\subsection*{WP3 Implementierung der App}

\subsubsection{Ziele:} Bedienung der Anwendung (aus der Sicht von Clients) ist über diverse Android-Geräte möglich.
\subsubsection{Beschreibung der Arbeit: [Wochen: 2-18]} Programmierung der App für Clients. Stetige Tests an verschiedenen aktuellen Android-Geräten.

\begin{enumerate}
\item [T3.1] API-Level festlegen (I) (W1, W1) (1 W): \emph{ Dadurch wird bestimmt, welche Andoid-Versionen mit dieser App kompatibel sind und welche Schnittstellen verwendet werden können.}
\item [T3.2] Aufteilung der Implementierung in Module (I) (W2, W3) (2 W): \emph{ Sinnvolle Aufteilunng der Implementationsschritte auf die verschiedenen Entwickler, damit eine parallele Abarbeitung möglich ist.}
\item [T3.3] Implementierung (I) (W17, W18) (2 W) : \emph{ Tatsächliche Implementierung der App mit sowohl kontinuierlichen als auch abschließenden Modultests.}
\item [T3.4] Zusammenfügen der Module (I) (W18, W18) (1 W) : \emph{ Module werden zur gesamten App zusammengefügt (inkl. abschliender Gesamttest).}
\item [T3.5] Verbindung zur Datenbank (I) (W18, W18) (1 W) : \emph{ Einrichtung der Schnittstelle zur Datenbank mit abschließendem Test.}
\item [T3.6] Dokumentation (I) (W18, W18) (1 W) : \emph{ Jeder Entwickler fertigt eine Dokumentation zu seinem Modul an, damit eine einfache Erweiterbarkeit bzw. Wartbarkeit möglich ist.}
\item [T3.7] Software-Wartung (I) (W19, W++) (XX W) : \emph{ Aktualisierung der Software für neue Android-Versionen.}
\end{enumerate}

\subsection*{WP4 Implementierung der Weboberfläche}

\subsubsection{Ziele:} Admins können über die Weboberfläche die Experimente erstellen und verwalten. Hierüber soll auch die Bedienung der App als Client möglich sein.
\subsubsection{Beschreibung der Arbeit: [Wochen: 7-36]} Entwicklung der App als Web-Anwendung.

\begin{enumerate}
\item [T4.1] Auswal eines passenden (PHP-)Frameworks (G) (W2, W3) (2 W): \emph{ Some short description (1-2 sentences).}
\item [T4.2] No-SQL Databases (G5, G6, G4, G1, G7, G4) (M7, M24) (10 M): \emph{ Some short description (1-2 sentences).}
\item [T4.3] Knowledge Bases and Data Mining (G5, G6, G1, G7, G8) (M7, M24) (14 M) : \emph{ Some short description (1-2 sentences).}
\item [T4.4] Ontology Services (G6, G5) (M7, M24) (10 M) : \emph{ Some short description (1-2 sentences).}
\item [T4.5] Processing and Analysis (G1, G4, G7, G5, G6, G10) (M7, M24) (8 M) : \emph{ Some short description (1-2 sentences).}
\item [T4.6] Predictive Models (G2, G7, G11, G6, G5, G9, G10) (M13, M30) (20 M): \emph{ Some short description (1-2 sentences).}
\item [T4.7] Workflows, Visualisation and Reporting (G10, G9, G6, G1, G2) (M13, M30) (14 M) : \emph{ Some short description (1-2 sentences).}
\end{enumerate}

\subsection*{WP5 Koordination und Management}

\subsubsection{Ziele:} Ziele hier einfügen
\subsubsection{Beschreibung der Arbeit: [Wochen: 1-36]} Beschreibung hier einfügen

\begin{enumerate}
\item [T5.1] Coordination \& Tracking (G1, G5, G6, G9, G3) (M1, M36) (11 M): \emph{ Some short description (1-2 sentences).}
\item [T5.2] General Assembly Meetings (G1, All) (M1, M36) (8 M) : \emph{ Some short description (1-2 sentences).}
\item [T5.3] Executive Board \& Advisory Board Meetings (G1, All) (M3, M36) (8 M): \emph{ Some short description (1-2 sentences).}
\item [T5.4] Reports (G1, All) (M10, M36) (9 M) : \emph{ Some short description (1-2 sentences).}
\end{enumerate}

%\newpage
\begin{landscape}
\begin{minipage}{\textwidth}
\begin{landscape}
\subsection*{Gantt Chart}
\begin{ganttchart}[hgrid, 
 vgrid, 
 bar label font =\small,
 bar/.append style={fill=red!50},
 y unit chart=0.5cm]{1}{18}
 \gantttitle{Dezember}{5}
 \gantttitle{Januar}{5}
 \gantttitle{Februar}{4}
 \gantttitle{Maerz}{4} \\

 \ganttbar{T1.1}{1}{3}\\
 \ganttbar{T1.2}{4}{12}\\
 \ganttbar{T1.3}{7}{18} 

\ganttnewline[thick]

 \ganttbar{T2.1}{1}{3} \\
 \ganttbar{T2.2}{1}{12}\\
 \ganttbar{T2.3}{1}{9}\\
 \ganttbar{T2.4}{1}{9}\\
 \ganttbar{T2.5}{1}{9}\\
 \ganttbar{T2.6}{7}{18}\\
 \ganttbar{T2.7}{7}{15}\\
 \ganttbar{T2.8}{7}{15}
\ganttnewline[thick]

 \ganttbar{T3.1}{4}{18} \\
 \ganttbar{T3.2}{7}{18}\\
 \ganttbar{T3.3}{4}{18}\\
 \ganttbar{T3.4}{7}{18}\\
 \ganttbar{T3.5}{1}{18}
\ganttnewline[thick]

 \ganttbar{T4.1}{4}{12} \\
 \ganttbar{T4.2}{4}{12}\\
 \ganttbar{T4.3}{4}{12}\\
 \ganttbar{T4.4}{4}{12}\\
 \ganttbar{T4.5}{4}{12}\\
 \ganttbar{T4.6}{4}{15}\\
 \ganttbar{T4.7}{6}{15}
\ganttnewline[thick]

 \ganttbar{T5.1}{1}{18} \\
 \ganttbar{T5.2}{1}{18}\\
 \ganttbar{T5.3}{2}{18}\\
 \ganttbar{T5.4}{5}{18}
\ganttnewline[thick]
\ganttmilestone{Milestones}{5}
\ganttmilestone{Milestones}{10}
\ganttmilestone{Milestones}{14}
\ganttmilestone{Milestones}{18}

\end{ganttchart}
\end{landscape}
\end{minipage}
\end{landscape}


\subsection*{Group Allocation Table}

Make up your own...

\end{document}
